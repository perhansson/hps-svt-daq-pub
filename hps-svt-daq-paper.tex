%% This is file `elsarticle-template-1-num.tex',
%%
%% Copyright 2009 Elsevier Ltd
%%
%% This file is part of the 'Elsarticle Bundle'.
%% ---------------------------------------------
%%
%% It may be distributed under the conditions of the LaTeX Project Public
%% License, either version 1.2 of this license or (at your option) any
%% later version.  The latest version of this license is in
%%    http://www.latex-project.org/lppl.txt
%% and version 1.2 or later is part of all distributions of LaTeX
%% version 1999/12/01 or later.
%%
%% The list of all files belonging to the 'Elsarticle Bundle' is
%% given in the file `manifest.txt'.
%%
%% Template article for Elsevier's document class `elsarticle'
%% with numbered style bibliographic references
%%
%% $Id: elsarticle-template-1-num.tex 149 2009-10-08 05:01:15Z rishi $
%% $URL: http://lenova.river-valley.com/svn/elsbst/trunk/elsarticle-template-1-num.tex $
%%
%%\documentclass[preprint,12pt]{elsarticle}

%% Use the option review to obtain double line spacing
%% \documentclass[preprint,review,12pt]{elsarticle}

%% Use the options 1p,twocolumn; 3p; 3p,twocolumn; 5p; or 5p,twocolumn
%% for a journal layout:
%% \documentclass[final,1p,times]{elsarticle}
%% \documentclass[final,1p,times,twocolumn]{elsarticle}
%% \documentclass[final,3p,times]{elsarticle}

 \documentclass[final,3p,times,twocolumn]{elsarticle}
% \documentclass[review,3p,times,twocolumn,sort&compress]{elsarticle}

%% \documentclass[final,5p,times]{elsarticle}
%% \documentclass[final,5p,times,twocolumn]{elsarticle}

%% if you use PostScript figures in your article
%% use the graphics package for simple commands
%% \usepackage{graphics}
%% or use the graphicx package for more complicated commands
%% \usepackage{graphicx}
%% or use the epsfig package if you prefer to use the old commands
%% \usepackage{epsfig}

%% The amssymb package provides various useful mathematical symbols
\usepackage{amssymb}
%% The amsthm package provides extended theorem environments
%% \usepackage{amsthm}

%% The lineno packages adds line numbers. Start line numbering with
%% \begin{linenumbers}, end it with \end{linenumbers}. Or switch it on
%% for the whole article with \linenumbers after \end{frontmatter}.
\usepackage{lineno}


%% natbib.sty is loaded by default. However, natbib options can be
%% provided with \biboptions{...} command. Following options are
%% valid:

%%   round  -  round parentheses are used (default)
%%   square -  square brackets are used   [option]
%%   curly  -  curly braces are used      {option}
%%   angle  -  angle brackets are used    <option>
%%   semicolon  -  multiple citations separated by semi-colon
%%   colon  - same as semicolon, an earlier confusion
%%   comma  -  separated by comma
%%   numbers-  selects numerical citations
%%   super  -  numerical citations as superscripts
%%   sort   -  sorts multiple citations according to order in ref. list
%%   sort&compress   -  like sort, but also compresses numerical citations
%%   compress - compresses without sorting
%%
%% \biboptions{comma,round}

% \biboptions{}

% HPS stuff
%\usepackage{url}
\usepackage{hyperref}
%\usepackage{fancyhdr}
\usepackage{color}
%\usepackage{multicol}
\newcommand{\Aprimebold}{\ensuremath{\mathrm{\mathbf{A}}^\prime}}
\newcommand{\Aprime}{\ensuremath{\mathrm{A}^\prime}}
%\newcommand{\Aprime}{A\ensuremath{^\prime}}
\newcommand{\ee}{e$^+$e$^-$}
\newcommand{\fluenceunit}{1~MeV~neutron~equivalent/cm\ensuremath{^2}}
\newcommand{\geant}{{\sc Geant4}}
\newcommand{\egs}{{\sc EGS5}}
\newcommand{\moliere}{Moli\`{e}re}
%\include{affiliations}

\journal{Nuclear Instruments and Methods in Physics Research Section A}

\begin{document}

\begin{frontmatter}


%% Title, authors and addresses

%% use the tnoteref command within \title for footnotes;
%% use the tnotetext command for the associated footnote;
%% use the fnref command within \author or \address for footnotes;
%% use the fntext command for the associated footnote;
%% use the corref command within \author for corresponding author footnotes;
%% use the cortext command for the associated footnote;
%% use the ead command for the email address,
%% and the form \ead[url] for the home page:
%%
%% \title{Title\tnoteref{label1}}
%% \tnotetext[label1]{}
%% \author{Name\corref{cor1}\fnref{label2}}
%% \ead{email address}
%% \ead[url]{home page}
%% \fntext[label2]{}
%% \cortext[cor1]{}
%% \address{Address\fnref{label3}}
%% \fntext[label3]{}



\title{The Data Acquisition System of the Silicon Vertex Tracker\\for the Heavy Photon Search Experiment}
%\title{
%\begin{flushleft}
%%       \mbox{\textmd{ \normalsize -PUB-13/020 }}
 %       \mbox{\textmd{ \normalsize SLAC-PUB-15999 }}
  %     \end{flushleft}
%{
%The Heavy Photon Search Test Detector
%}
%}

%%%%%%%%%%%%%%%%%%%%%%%%%%%%%%%%%%%%%%%%%%%%%%%%%%
%% use optional labels to link authors explicitly to addresses:
%% \author[label1,label2]{<author name>}
%% \address[label1]{<address>}
%% \address[label2]{<address>}

\input{author_list}


%\author[slac]{Per Hansson Adrian\corref{cor1}}
%\ead{phansson@slac.stanford.edu}
%\address[slac]{SLAC National Accelerator Laboratory, 94025, Menlo Park, CA, USA}

%%%%%%%%%%%%%%%%%%%%%%%%%%%%%%%%%%%%%%%%%%%%%%%%%%


\begin{abstract}
The Heavy Photon Search (HPS) experiment at the Thomas Jefferson National Accelerator Facility (JLab) search for heavy photons which are new 
hypothesized massive vector bosons. The experiment consists of an electromagnetic calorimeter for triggering and particle identification and a 
a Silicon Vertex Tracker (SVT) for momentum  and vertex position measurements. The data acquisition system (DAQ) for the SVT supports readout and 
processing of signals from 36 silicon strip sensors of the SVT. It also selects and transfers those events that were identified by the trigger system to the 
JLab DAQ for further event processing at rates approaching 50~kHz. Complex front-end electronics for digitization and power distribution are deployed 
inside the SVT vacuum box in order to reduce vacuum penetration count. A strong magnetic field used by the experiment also complicates the 
electronics design. The SLAC RCE Platform is utilized upstream to process and filter raw sensor data for delivery to the JLab DAQ. The system is 
currently deployed and running.
\end{abstract}

\begin{keyword}
%% keywords here, in the form: keyword \sep keyword
data acquisition \sep atca \sep silicon \sep heavy photon \sep dark photon 
%% MSC codes here, in the form: \MSC code \sep code
%% or \MSC[2008] code \sep code (2000 is the default)

\end{keyword}


\end{frontmatter}

 %\linenumbers
 
\tableofcontents
%\clearpage
%\newpage
\newpage

%%
%% Start line numbering here if you want
%%
%\linenumbers

%% main text

\section{Introduction}
% The very first letter is a 2 line initial drop letter followed
% by the rest of the first word in caps.
% 
% form to use if the first word consists of a single letter:
% \IEEEPARstart{A}{demo} file is ....
% 
% form to use if you need the single drop letter followed by
% normal text (unknown if ever used by IEEE):
% \IEEEPARstart{A}{}demo file is ....
% 
% Some journals put the first two words in caps:
% \IEEEPARstart{T}{his demo} file is ....
% 
% Here we have the typical use of a "T" for an initial drop letter
% and "HIS" in caps to complete the first word.

%%%%%%%%%%%%%%%%%%%%%%%%%%%%%%%%%%%%%%%%%%%%%%%%%%%%%%%%%%%%%%%%%%%%%%%%

The Heavy Photon Search experiment (HPS) is a new fixed-target experiment~\cite{proposal_full}
specifically designed to discover an \Aprime{} with m$_{\Aprime}=20-500$~MeV, produced through bremsstrahlung 
in a tungsten target and decaying into an $e^{+}e^{-}$ pair.  
In particular, the HPS experiment has sensitivity to the challenging region with small cross sections out of 
reach from collider experiments and where thick absorbers, as used in beam-dump experiments to 
reject backgrounds, are ineffective due to the relatively short \Aprime{} decay length ($<1$~m)~\cite{bible}.  
This is accomplished 
by placing a compact silicon tracking and vertex detector (SVT) in a magnetic field, immediately downstream (10~cm) 
of a thin ($\sim 0.125\%~X_{0} $) target to reconstruct the mass and decay vertex position of  the \Aprime{}. A 
rendered overview of HPS is shown in Fig.~\ref{fig:hps-layout}.
\begin{figure}[]
\centering
\includegraphics[width=3.5in]{figures/hps-layout-cropped}
\caption{A rendered overview of the HPS detector.}
\label{fig:hps-layout}
\end{figure}

HPS runs in Hall~B at Thomas Jefferson National Accelerator Facility (JLab) using the CEBAF 
accelerator electron beam with an energy of 1.05~GeV and 50~nA current, with planned operation of up to 6.6~GeV and 450~nA. 
The kinematics of \Aprime{} production 
typically result in final state particles within a few degrees of the beam, especially at low $m_{\Aprime}$ . 
Because of this, the detector must accommodate passage of the beam 
downstream of the target and operate as close to the beam as possible. Because background rates in this region from the 
scattered beam are very large, a fast lead-tungstate crystal calorimeter trigger with 250~MHz flash ADC readout~\cite{fadc250} and 
excellent time tagging of hits is used to trigger on interesting events and reduce the bandwidth required to transfer data from the 
detector.  This method of background reduction is the motivation for operating HPS in a
nearly continuous beam: in a beam with large per-bunch charge, background from a single bunch would fully occupy the detector 
at the required beam intensity. 


The SVT  comprises 36 silicon strip sensors, each attached to a hybrid board incorporating five 128-channel APV25 front-end ASICs. 
The APV25 ASIC was initially developed for the 
Compact Muon Solenoid silicon tracker at the Large Hadron Collider at CERN~\cite{apv25_2}. 
Each hybrid board has five analog output lines (one for each of the APV25 ASICs) which are sent to Front End Boards (FEBs) using low power LVDS 
differential current sig-nals over about 1 m of twisted pair cable. At 
each FEB, a pre-amplifier scales the APV25 differential current output to match the range of a 14-bit Analog to Digital Converter (ADC). Both the ADCs 
and the APV25s operate at the system clock frequency of 41.667~MHz. There are 10 FEBs deployed in the SVT, each serving up to four hybrids. Each 
FEB also distributes low voltage power and high voltage bias to each attached hybrid. The deployment of readout electronics within the experiment 
vacuum chamber and 2 T magnetic field presents many challenges for the DAQ system design. The SLAC RCE general purpose DAQ System [3] is 
utilized on the back end for data filtering, formatting, and transmission for storage. An electro-magnetic calorimeter downstream of the SVT issues 
readout triggers through the JLab timing system


There should be a list of requirements and design goal for the DAQ listed below. These can be addressed and referred to in the later sections.
\begin{itemize}
\item S/N, number of channels
\item time resolution
\item trigger rate
\item data rate
\item integration with JLab DAQ
\item radiation hardness
\item operation, maintenance and longevity
\end{itemize}


%%%%%%%%%%%%%%%%%%%%%%%%%%%%%%%%%%%%%%%%%%%%%%%%%%%%%%%%%%%%%%%%%%%%%%%%

\section{The Heavy Photon Search Experiment}
\label{sec:hps}
\subsection{Electromagnetic Calorimeter Trigger}
\label{sec:ecal}
\subsection{The Silicon Vertex Tracking Detector}
\label{sec:svt}
The Silicon Vertex Tracker (SVT) allows for precise and efficient reconstruction of charged particles and their trajectories. 
%The design was driven mainly by the specific physics signature of \Aprime{} production and decay and the constraints from the environment 
%at the interaction region. 
At beam energies between 1.0-6.6~GeV, the electron and positron from the \Aprime{}  
decay will be produced with momenta in the range of 0.4-2~GeV$/c$ and angles of 10-100~mrad from the beam. The dominant 
tracking uncertainty in this regime is multiple Coulomb scattering, so the SVT needs to minimize the amount of material in the tracking 
volume. With an approximate goal of 2\% mass resolution in the 1~m long tracking volume (determined by an existing magnet and vacuum 
chamber) with 0.25~T magnetic field (for 1~GeV beam energy); 1\%~$X_0$ or less material per 3D tracking hit and 
six layers was deemed adequate. 
For weak couplings, the \Aprime{} may be long-lived and the $e^{+}e^{-}$ pair decay vertex might be displaced several 
cm downstream of the target foil. To discover rare \Aprime{} displaced decays, the SVT typically needs a prompt rejection of roughly 
10$^7$ at 1~cm vertex resolution~\cite{proposal_full}. In order to reach that performance, the first layer of the SVT needs to be placed 
10~cm from the target. At that distance, the large hit rates from beam electrons undergoing Coulomb scattering in the target allow placing 
the first layer 1.5~mm from the beam. No instrumentation can be placed inside that 15~mrad angle creating a "dead zone" throughout the 
experiment. The expected radiation dose peaks at $10^{15}$~electrons/cm$^2$/month, or roughly 
$3 \times 10^{13}$~\fluenceunit{}/month~\cite{dose}, close to the beam and places further constraints on the sensor 
technology.  Furthermore, the whole tracker has to operate in vacuum to avoid secondary backgrounds from 
beam gas interactions, and have retractable tracking planes and easy access for sensor replacement to increase safety. 
Given the high hit density, the fast time response, and good resolution and radiation hardness needed; silicon microstrip 
sensors are the technology of choice for the tracker. Pixel sensors suitable for instrumenting our large acceptance are either too slow
or have an unacceptable material budget.


The SVT overall layout is rendered in Fig.~\ref{fig:layout} and summarized in Tab.~\ref{tab:layout}. 
\begin{figure}[]
\centering
\includegraphics[width=3.5in]{figures/svt-layout.png}
\caption{A rendered overview of the SVT installed on the beamline.}
\label{fig:layout}
\end{figure}
\begin{table}
\centering
\begin{tabular}{|lcc|}
\hline
Layer $\rightarrow$& 1-3 & 4-6 \\ 
\hline
$z$ pos. (cm)  & 10-30 & 50-90  \\
Stereo angle (mrad)  & 100 & 50  \\
Non-bend plane resolution ($\mu$m)  & $\approx 6$ & $\approx6$  \\
Bend plane resolution ($\mu$m)  & $\approx 60$ & $\approx 120$  \\
\hline
\end{tabular}
\caption{Main tracker parameters.}
\label{tab:layout}
\end{table}
Each of the six tracking layers, 
arranged in two halves both above and below the beam to avoid the "dead zone", consists of silicon microstrip sensors placed back-to-back. 
The first three layers have a 100~mrad stereo angle between the sensors and the last three have 50mrad in order to improve the pointing 
resolution to the vertex. The first layer is located only 10~cm downstream of the target to give excellent 3D vertexing performance which, 
with the 15~mrad dead zone above and below the beam axis, puts the active silicon only 1.5~mm from the center of the beam. Hit densities 
in the most active region reach 4~MHz/mm$^2$ and about 1\% occupancy for the strips closest to the beam, see Fig.~\ref{fig:occupancy}.
\begin{figure}[]
\centering
\includegraphics[width=3in]{figures/occupancy.png}
\caption{Background occupancy for layer 1 during nominal operation conditions.}
\label{fig:occupancy}
\end{figure}





%%%%%%%%%%%%%%%%%%%%%%%%%%%%%%%%%%%%%%%%%%%%%%%%%%%%%%%%%%%%%%%%%%%%%%%%

\section{Data Acquisition System}

Discussion about the specific requirements that the DAQ have to fulfill. The description of HPS and the SVT should be enough to be able to 
give this section a solid context.


%%%%%%%%%%%%%%%%%%%%%%%%%%%%%%%%%%%%%%%%%%%%%%%%%%%%%%%%%%%%%%%%%%%%%%%%

\subsection{Overview}

Analog samples at 41.667~MSPS from the APV25 chips are sent on twisted pair magnet wire 
to a total of 10 Front End Boards (FEB) seen in Fig.~\ref{fig:febs}. Each FEB digitizes and transfers data, from up to four hybrids, 
at up to 3.3~Gb/s using high-speed serial links to Xilinx Zynq based data processing modules on the ATCA based SLAC RCE 
platform~\cite{paper_testrun} for zero suppression and event building.
Each FEB also handles power regulation and monitoring as well as high voltage sensor bias distribution to each of the attached 
hybrids. To shorten the analog signal distance, the FEBs are placed inside the vacuum chamber, pressed against thermal pads on each side of a 1/2$"$ cooled support plate on the 
upstream positron side, rendering a less intense radiation environment. Borated high-density polyethylene is used to 
further lower the risk of damage from radiation emitted by the nearby target.
Data from the FEBs are routed via short, flexible miniSAS cables to four flange boards. These are FR4 boards potted through slots in the 
8" vacuum flange on the upstream positron side of the vacuum box. On the out side of the boards, signals are converted to optical and transferred to 
the DAQ platform about 30~m away. A similar mechanical technique is used on the opposite side of the vacuum box to bring in low voltage power 
and high voltage sensor bias into the chamber; this can be seen in Fig.~\ref{fig:vacuum-box}.

An overview of the data flow across the RCE platform is shown in Fig.~\ref{fig:rce}.
\begin{figure}[]
\centering
\includegraphics[width=3.5in]{figures/rce.png}
\caption{Schematic overview of the SVT DAQ.}
\label{fig:rce}
\end{figure}
Data from 10 FEBs are split and sent to 14 processing nodes on two ATCA blades, called COBs (Cluster on board). 
The processing nodes, known as Reconfigurable Cluster Elements (RCE) are based on Xilinx Zync 7000 series 
system-on-chip which has a dual ARM Cortex A9 processor tightly coupled to a 28nm FPGA fabric. The independent operating nodes receive data 
from up to four hybrids, apply calibrated thresholds and build event frames in the firmware. A readout application from the JLab DAQ~\cite{coda}, 
running on the ARM processor, pulls the event frames from memory via DMA and transfers the event frames to the JLab DAQ 
event builder over 10~Gb/s ethernet. 
The COB also hosts a special RCE that handles the trigger and timing distribution across the processing nodes on each COB. 
This RCE implements the JLab trigger interface 
firmware and accepts the master clocks together with trigger information from the JLab DAQ from a special fiber attached on the 
custom rear transition board.  One of the RCEs is allocated to handle control, trigger and timing signals to and from all the hybrids. It also hosts the 
slow control and environmental interfaces to the EPICS control system. 

During running the system operated at about 20~kHz and with data rates up to 150~MB/s. It has been tested to 50~kHz and 200~MB/s.



\subsection{Sensors and Front-End Readout}
\label{sec:apv25}
The sensors are $p+$-on-$n$, single-sided, AC-coupled, polysilicon-biased microstrip sensors fabricated by Hamamatsu Photonics Corporation 
for the cancelled D\O~Run~2b upgrade~\cite{d0run2b}. These $320~\mu$m thick sensors are $4\times10$~cm$^2$ with 30 and 60~$\mu$m pitch 
for sense and readout strips, respectively,  matching the required material budget and single hit spatial resolution. 
The sensors were qualified to withstand at least 1~kV bias in order to tolerate the $1.5\times10^{14}$~\fluenceunit{} for a 
six month run without significant degradation. 

One of the key requirements for the SVT is hit time resolution of  $<2$~ns in order to reject background and improve pattern recognition 
accuracy for close to the beam where occupancies are high. This is achieved by using the APV25 front-end readout ASIC~\cite{apv25},  
developed for the 
Compact Muon Solenoid experiment at the Large Hadron Collider. The APV25 is an analog pipeline ASIC with 128 channels of preamplifier 
and shaper, feeding a 192 long analog memory pipeline. In the so-called "multi-peak" readout mode, the APV25 presents 
three consecutive samples of the pulse height in response to an APV25 readout trigger signal. 
By sending two APV25 readout triggers for every event trigger signal 
from the electromagnetic calorimeter, six analog samples of the pulse shape, see Fig.~\ref{fig:pulseshape}, are obtained at a sampling rate of 
41~MSPS. This pulse shape can be analyzed and fitted to extract the t$_0$ of the hit~\cite{Friedl:2009zz}.
\begin{figure}[]
\centering
\includegraphics[width=3.0in]{figures/pulseshape.png}
\caption{Pulse shape of the APV25 ASIC.}
\label{fig:pulseshape}
\end{figure}
The main parameters of the APV25 ASIC are shown in Tab.~\ref{tab:apv25}: the 44~$\mu$m pitch, low noise and operation using either polarity together with the 
proven robustness and radiation hardness is a good fit for HPS. 
\begin{table}
\centering
\begin{tabular}{|l|c|}
\hline
Technology & 0.25~$\mu$m \\ 
\hline
Channels & 128 \\
\hline
Input pitch & 44~$\mu$m \\
\hline
Noise [ENC e$^-$] & $270 + 36\times$C~(pF)\\
\hline
Power consumption & 350mW \\
\hline
\end{tabular}
\caption{Main APV25 ASIC parameters.}
\label{tab:apv25}
\end{table}
The sensor and APV25 chip can be seen in Figure~\ref{fig:half-modules-L1-3}.


\subsection{Hybrid}
\label{sec:hybrid}
The SVT modules are built by placing two identical, so-called half-modules back-to-back at a stereo angle. Each half-module is built by gluing silicon 
sensors and an accompanied hybrid readout board to a polyimide-laminated carbon fiber frame. The first three layers of the SVT inherits the half-module from the HPS Test detector~\cite{paper_testrun}, shown in Fig.~\ref{fig:half-modules-L1-3} while the last three layers are larger to increase 
tracking acceptance as shown in  Fig.~\ref{fig:half-modules-L4-6}.
\begin{figure}[]
\centering
\includegraphics[width=3.0in]{figures/halfmodule-L1-3.jpg}
\caption{A half-module for layer 1-3 mounted on the module support.}
\label{fig:half-modules-L1-3}
\end{figure}
\begin{figure}[]
\centering
\includegraphics[width=3.0in]{figures/halfmodule-L4-6-assembly.jpg}
\caption{A layer 4-6 half-module under assembly.}
\label{fig:half-modules-L4-6}
\end{figure}
The hybrid board caries five APV25 ASICs that are wire-bonded to 128 silicon strip lines each, except for the last channel on one of the outermost 
chips, for a total of 639 sensor lines. There are 36 hybrid boards in the SVT, and therefore 23,004 readout channels in total. The APV25 samples each 
of the attached channels at 41.667~MHz into a 192 entry analog memory pipeline. 
Upon reception of a system trigger signal, samples for each channel are read out from a preselected pipeline depth. The readout frame consists of 
a $\pm~4$~mA rail-to-rail digital header followed by an analog differential current representing each channel. The APV25 is configured in "multi-peak" 
mode, so that three consecutive frames are output per trigger. Hit time reconstruction with 2~ns resolution requires six consecutive samples, thus two 
APV25 triggers are issued back-to-back in response to each DAQ system trigger. Figure~\ref{fig:pulseshape} shows the measured pulse shape.  
\begin{figure}[]
\centering
\includegraphics[width=3.0in]{figures/pulseshape.png}
\caption{Pulse shape of the APV25 ASIC.}
\label{fig:pulseshape}
\end{figure}
A six sample trigger readout takes 20.1~$\mu$s. This limits the 
maximum trigger rate to 49.7~kHz. The APV25 can accept new triggers before the previous trigger data are read out. Up to five triggers may be 
�stacked� in this manner.  


The APV25 chips are configured via an I$^2$C interface. An additional I$^2$C temperature monitor is also present on each hybrid to monitor that the 
silicon sensors are being properly cooled. Each hybrid requires 2.5~V (AVDD) and 1.25~V references to power the APV25s, and an additional 2.5~V 
(DVDD) source to power digital components such as the I$^2$C temperature monitor and differential clock and trigger fanouts. High voltage (<1~kV) 
bias for the silicon detectors strips is routed directly to wire bond pads with local bypassing.


\subsection{Front End Board}
\label{sec:feb}
The FEB serves two purposes: distribute power to the hybrids and digitize analog readout data from each hybrid APV25. Its design and placement 
inside the SVT vacuum chamber is motivated by a desire to reduce vacuum penetration count and analog signal length. The design centers around a 
Xilinx Artix-7 FPGA to interface to the ADCs, transmit digitized data up-stream, distribute clock, trigger, and I$^2$C communication to the hybrids, and 
control and monitor hybrid power distribution.
Deployment of FEBs inside the SVT chamber presents a number of challenges. Due to the presence of a 2~T magnetic field, great care was taken with 
the design of all onboard power regulators. Ferrite core inductors commonly used with DC-DC switching regulators will saturate in such high fields and 
lose effectiveness. It was also desired that no magnetically conduct-ing material be used on the board, so as not to distort the magnetic field inside the 
chamber. This necessitated the use air-core inductors in power regulation circuits and throughout the board.
Radiation is also a concern inside of the SVT chamber. Beam interactions with the Tungsten target produce both neutrons and x-rays, which can 
have adverse effects on the FEBs. Neutrons can cause Single Event Upsets (SEUs) in the digital circuits of the FPGA, and x-ray doses can degrade 
integrated circuits over time. Simulations indicated that these sources should be within acceptable limits, but additional measures were taken just in 
case. A Borated-Polyethylene shield was installed around the FEBs to block neutrons, and the boards are installed on a serviceable custom cooling 
plate so that they can be replaced if needed.
\begin{figure}[]
\centering
\includegraphics[width=3.0in]{figures/feb.png}
\caption{One COB ATCA blade used in the RCE platform.}
\label{fig:cob}
\end{figure}
Each FEB connects to 4 hybrids, for a total of 20 APV25 analog channels. A preamplifier circuit converts the �4 mA signal from each APV25 into a 
voltage scaled to the range of the AD9252 14-bit ADC. The ADC sample clock runs at the same frequency as the APV25 clock, but has a 
programmable phase offset so that each sample can be taken at the center-point be-tween analog transitions.
The FPGA monitors each APV25 ADC stream looking for readout frames. Every readout frame is then sent upstream by a multi-gigabit transceiver 
(MGT). The FEB has four upstream-only MGTs, with one dedicated to each hybrid. By dropping the two least significant noise bits from each ADC 
sample, each hybrid�s readout data can be packed on to a 3.125 Gbps link at rates approaching 50~kHz. Performing APV25 frame recovery on the 
FPGA in this manner allows the link speed to scale with the trigger rate, as well as robust error recovery on the upstream end. At the maximum trigger 
rate of 48.7~kHz, the combined data output rate from all of the FEBs is 89.6 Gbps.
An additional full duplex MGT provides for configuration, trigger, and clock to be received from the upstream system. This link operates in a special 
fixed-latency mode so that every FEB in the system can recover the same 125~MHz beam-synchronous clock with minimal skew. The recovered clock is 
divided on each FEB to create the 41.667~MHz APV25 and ADC clocks. Triggers and clock alignment commands can be sent down these links with a 
guaranteed latency, assuring that all APV25s in the system receive the same clock and triggers in lockstep with each other.
The FEB is also responsible for distributing and monitoring hybrid power. Switched-mode regulators are used to efficiently step down a 6V reference to 
three intermediate voltages � 2.9V, 2.9V and 1.4V. Linear regulators then convert these voltages into the 2.5V (DVDD), 2.5V (AVDD) and 1.25V 
needed by each hybrid. AD5144 SPI digital potentiometers placed in the resistor feedback of each regulator allow for all of the regulated voltages to be 
trimmed by the FPGA to account for cable drops. LTC2991 I$^2$C ADCs are deployed to monitor the output voltage, feedback voltage and output 
current on each of the twelve hybrid voltage outputs. All of these monitors are accessible on the control link, and are sampled every second for delivery 
to EPICS slow controls.
\begin{figure}[]
\centering
\includegraphics[width=3.0in]{figures/febs.jpg}
\caption{The partially cabled data acquisition front end boards  screwed to the aluminum support plate before installation .}
\label{fig:febs}
\end{figure}



\subsection{Flange Board}
\label{sec:flange}
High speed differential signals from the FEB MGTs are sent through compact 8-pair mini-SAS cables to custom built vacuum flange board as seen 
in Fig.~\ref{fig:flange-board}.
\begin{figure}[]
\centering
\includegraphics[width=3.0in]{figures/flange-board.png}
\caption{Flange board.}
\label{fig:flange-board}
\end{figure}
Each of these boards is potted into the flange across its midpoint, with optical conversion of the differential signals occurring on the outside portion of 
the board. The optical signals are then transmitted over 30~m fibers to the SLAC RCE crate. Each of these flange boards can connect up to three FEBs, 
so four identical boards are potted into a single flange to handle all 10 FEBs. Separate custom flange boards carry low voltage power and high voltage 
bias through the vacuum penetration to the FEBs. A picture of the installed boards is shown in Fig.~\ref{fig:vacuum-box}.
\begin{figure}[]
\centering
\includegraphics[width=3.0in]{figures/vacuum-box.jpg}
\caption{View from upstream, electron side, of the SVT after installation on the beamline. The vacuum box, that interfaces the beamline with the 
vacuum chamber, can be seen with its three linear shifts, two on the top and one on the bottom side. The flange holding the power and 
high-voltage sensor bias vacuum penetration boards extends to the right from the support box and the high-speed signal flange boards are attached 
to the opposite side.}
\label{fig:vacuum-box}
\end{figure}


\subsection {RCE Platform}
\label{sec:rce}

\subsubsection{Overview}
The SVT DAQ uses the SLAC RCE Platform for high speed data transfer. This is a general purpose data processing cluster based on the Xilinx 
Zynq-7000 SOC and ATCA backplane. Each ATCA blade is called a Cluster on Board (COB) and contains eight Zynq-7040 nodes (RCEs) for data 
processing and one Zynq-7020 node for timing and trigger distribution (DTM). Each of these nodes has a dual-core ARM CPU with 1~GB of DDR3 
memory tightly integrated with on-chip programmable logic (FPGA). A Rear Transition Module (RTM) specific to each experiment distributes high 
speed serial links to the RCEs and DTM. An Ethernet switch connects each node to the ATCA backplane and to front panel SFPs for upstream 
connectivity. Figure~\ref{fig:cob} shows a typical COB used in the RCE platform. 
\begin{figure}[]
\centering
\includegraphics[width=3.0in]{figures/cob.png}
\caption{One COB ATCA blade used in the RCE platform.}
\label{fig:cob}
\end{figure}

The RCE configuration for the HPS SVT DAQ is shown in Fig.~\ref{fig:rce-hps-detailed}. 
\begin{figure}[]
\centering
\includegraphics[width=3.0in]{figures/rce-layout.png}
\caption{Block diagram of the RCE layout for the HPS experiment.}
\label{fig:rce-hps-detailed}
\end{figure}

\subsubsection{Layout and Data Flow}

Two COBs are deployed for a total of 16 RCE nodes. Optical signals from the vacuum flanges arrive at the RTMs and are converted to differential 
signals for the RCEs. Each RCE processes up to four high speed links, with each link containing APV25 readout data from one hybrid. One RCE is 
configured as the exclusive FEB control node. It provides fixed-latency MGT control links at 2.5 Gbps to each of the 10 FEBs through the RTM. These 
links carry configuration, clock and trigger data.

The RTM also provides an interface to the JLab timing system for the DTM. FPGA logic in the DTM implements a full JLab Trigger Interface (TI), 
eliminating the need for a discrete TI card. Clock and trigger data are decoded by the TI logic and distributed to each of the DPMs via COB 
interconnects. Each RCE has an interface back to the DTM to indicate how many trigger events it has processed. The TI logic then communicates this 
back to the JLab Trigger System.

All other RCE are configured for data processing. Each of these nodes receives the digitized APV25 readouts from each trigger, applies thresholds for 
data reduction, and puts the data into RAM via DMA transfers. Software running on the Zynq ARM CPU then organizes the data into Ethernet frames 
and sends them to the higher level JLab DAQ system described in Sec.~\ref{sec:roc}.

\subsubsection{Data Processing}
Detailed discussion of the data processing on the data DPM's.

\subsubsection{Timing Distribution}
Detailed discussion on the timing and trigger distribition on the COB. Note that Sec.~\ref{sec:integration} talks in detail about integration with JLab.

\subsubsection{Control and Monitoring}
Detailed description on how the system is controlled and monitored. Servers running on each DPM.  Specifically how the control DPM monitor and control hybrids. 


\section{Integration with the JLab Data Acquisition}

\subsection{Overview}
Give general context of the JLab DAQ to the extent that the integration discussion below have some context.

\subsection{Trigger and Timing Interface}
The RTM  provides an interface to the JLab timing system for the DTM. FPGA logic in the DTM implements a full JLab Trigger Interface (TI), 
eliminating the need for a discrete TI card. Clock and trigger data are decoded by the TI logic and distributed to each of the DPMs via COB 
interconnects. Each RCE has an interface back to the DTM to indicate how many trigger events it has processed. The TI logic then communicates this 
back to the JLab Trigger System.

\subsection{Event Building}
\label{sec:roc}
Description of the data flow and how events are built and transferred from the firmware to the ROC and onwards to the CODA event builder.

\subsection{Operation}
Description of software including GUIs and EPICS interface that us used to operate the DAQ. 

\section{Performance}

Define what belongs here, examples:
\begin{itemize}
\item data and trigger rates
\item dead/bad channels, signal to noise
\item hit time resolution, 
\item APV25 issues (occupancy dependent baseline, header noise)
\item SEU candidate event
\end{itemize}



\section{Summary}

HPS completed a successful two week engineering run in the Spring of 2015. Nominal trigger rates of up to 20~kHz were demonstrated, with data rates 
of up to 150~MB/s. Trigger rates up to 47~kHz have been demonstrated offline. Additional enhancements are currently being implemented to 
reduce dead-time by fully utilizing of the APV25 output pipeline. Another data run is expected in the Spring of 2016.

















\section{Acknowledgements}
The authors are grateful for the support from Hall~B at JLab and especially the Hall~B engineering 
group for support during installation and decommissioning. They also would like to commend the 
CEBAF personnel for good beam performance, especially the last few hours of operating CEBAF6. 
The tremendous support from home institutions and supporting staff also needs praise from the 
authors. 

Work supported by the U.S. Department of Energy under contract number DE-AC02-76SF00515 and the National Science Foundation.


%% The Appendices part is started with the command \appendix;
%% appendix sections are then done as normal sections
%% \appendix

%% \section{}
%% \label{}

%% References
%%
%% Following citation commands can be used in the body text:
%% Usage of \cite is as follows:
%%   \cite{key}          ==>>  [#]
%%   \cite[chap. 2]{key} ==>>  [#, chap. 2]
%%   \citet{key}         ==>>  Author [#]

%% References with bibTeX database:

\bibliographystyle{model1-num-names}
%\bibliography{<your-bib-database>}
\bibliography{hps-svt-daq-paper}

%% Authors are advised to submit their bibtex database files. They are
%% requested to list a bibtex style file in the manuscript if they do
%% not want to use model1-num-names.bst.

%% References without bibTeX database:

% \begin{thebibliography}{00}

%% \bibitem must have the following form:
%%   \bibitem{key}...
%%

% \bibitem{}

% \end{thebibliography}


\end{document}

%%
%% End of file `elsarticle-template-1-num.tex'.
